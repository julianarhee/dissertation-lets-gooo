% the abstract

Vision is critical for guiding behavior in species throughout the animal kingdom. Nonhuman primates have long been classic models for vision, but over the last decade, rodents have emerged as highly promising alternatives. Among rodents, rats are broadly favored for their rich behavioral repertoire and capacity for learning. In the context of vision, rats have higher visual acuity than mice, rely on vision for spatial navigation, and can even be found hunting during the day. However, they pose significant challenges for chronic, cellular resolution imaging in awake, behaving animals. This limits the extent to which existing molecular and genetic tools can be leveraged with optical imaging in rats, as they have been with great success in smaller model organisms, such as mice or flies. Here, we overcome these challenges and characterize visual response properties for a range of simple and complex stimuli across multiple cortical areas in awake, head-restrained rats. 

We first train rats on a task of invariant object recognition, a complex visual capacity that refers to the ability to recognize a given object across the many ways it can appear. We show that rats demonstrate robust, spontaneous generalization across both identity-preserving and identity-varying transformations of complex object shapes. Using the same stimuli, we then characterize neuronal populations in naive, head-fixed animals using cellular resolution imaging, and find intrinsic generalization capacities across multiple areas of lateral visual cortex. The present study demonstrates the feasibility of chronic, cellular resolution imaging in behaving rat models, and highlights the utility of studying visual processes like object recognition across different species.
