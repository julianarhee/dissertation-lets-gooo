% the abstract

Vision is critical for guiding behavior in species throughout the animal kingdom. Nonhuman primates have long been classic models for vision, but over the last decade, rodents have emerged as highly promising alternatives. Among rodents, rats are broadly favored for their rich behavioral repertoire and cognitive capacities. In the context of vision, rats have higher visual acuity than mice, rely on vision for spatial navigation, and can even be found hunting during the day. However, they pose significant challenges for chronic, cellular resolution imaging in awake animals. This limits the extent to which existing molecular and genetic tools can be leveraged in rats, as they have been with great success in smaller model organisms, such as mice. Here, we overcome these challenges and characterize visual response properties across multiple cortical areas in awake, head-restrained rats. 

We first train rats on a task of invariant object recognition, a complex visual capacity that refers to the ability to recognize a given object across the many ways it can appear. Rats demonstrate robust, spontaneous generalization across both identity-preserving and identity-varying transformations of visual objects. Using the same stimuli, I then characterize neuronal populations in naive, head-fixed rats using cellular resolution imaging, and describe response properties to both simple and complex visual stimuli across multiple areas of lateral visual cortex. The present study demonstrates the feasibility of chronic, cellular resolution imaging in the behaving rat, a powerful animal model for cognitive behaviors, and highlights the utility of studying neural processing and behavior across different animal species.
