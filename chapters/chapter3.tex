\begin{savequote}[75mm]
Nulla facilisi. In vel sem. Morbi id urna in diam dignissim feugiat. Proin molestie tortor eu velit. Aliquam erat volutpat. Nullam ultrices, diam tempus vulputate egestas, eros pede varius leo.
\qauthor{Quoteauthor Lastname}
\end{savequote}

\chapter{Basic characterizations}
% Figure:  RF mapping, examples
% Figure:  RFs, aggregate/summary plots
% Figure:  GRATINGS, stimuli + example traces
% Figure:  Summary stats of OSI, DSI, "Fraction tuned"
% Figure:  RFs + ORI preferences

\newthought{Lorem ipsum dolor sit amet}, consectetuer adipiscing elit. Morbi commodo, ipsum sed pharetra gravida, orci magna rhoncus neque, id pulvinar odio lorem non turpis. Nullam sit amet enim. Suspendisse id velit vitae ligula volutpat condimentum. Aliquam erat volutpat. Sed quis velit. Nulla facilisi. Nulla libero. Vivamus pharetra posuere sapien. Nam consectetuer. Sed aliquam, nunc eget euismod ullamcorper, lectus nunc ullamcorper orci, fermentum bibendum enim nibh eget ipsum. Donec porttitor ligula eu dolor. Maecenas vitae nulla consequat libero cursus venenatis. Nam magna enim, accumsan eu, blandit sed, blandit a, eros.



\section{Lateral visual cortex exhibits many of the same core properties found in primates}

Since we have demonstrated the key prerequisite of establishing the behavioral ability of rats to perform visual recognition has been established, we next turned our attention to the neuronal substrates of these abilities.  Primate visual cortex is arranged hierarchically, with visual inputs from the thalamus first arriving in so-called ``striate'' cortex (also known as area V1), before being processed and forwarded through a successive chain of hierarchically-organized visual areas (area V2 > area V4 > inferotemporal cortex) curving along the ventral surface of the brain.  

Several key trends have been observed in the response properties of visual neurons as one progresses from ``lower'' to ``higher'' visual areas along this ventral pathway. First, the region of visual space that a given cell responds to (the ``receptive field'') gradually increases as one moves along the ventral pathway, with receptive fields in the highest stages of visual cortex sometimes responding to up to half of the visual field \cite{op2000spatial}. Meanwhile, selectivity for complex object features also increases along the ventral visual pathway, with neurons in later stages of the pathway responding only to very particular configurations of features \cite{Desimone1984, Logothetis1996}.  Critically, as one progresses along the ventral pathway, neurons also exhibit greater tolerance to identity-preserving transformation of the retinal image -- that is, neurons tend to retain their selectivity for particular object features even if those features are, for instance, moved around on the retina, or scaled up or down in size \cite{Ito1995}.These combined features of selectivity and tolerance are in many ways the key computational hallmarks of high-level vision \cite{DiCarlo2007, DiCarlo2012}. 

Anatomical studies have shown that the connectivity of rodent visual cortex observes a similar hierarchical pattern, with thalamic inputs arriving in an analogous striate area V1 in posterior of the brain, and then projecting ventrally to a series of interconnected extrastriate areas  \cite{Coogan1993, ETC}.  However, while these areas have been characterized anatomically, very little is known about their function.

