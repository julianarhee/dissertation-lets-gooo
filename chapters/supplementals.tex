\chapter{Supplemental Materials}
\label{supplementals}

To do.

% Software list/table
% DeepLabCut \href{https://github.com/DeepLabCut/DeepLabCut}
% MWorks \href{https://mworks.github.io}
% ScanImage 2016 \href{http://scanimage.vidriotechnologies.com}

% Custom software
\href{https://github.com/coxlab/behavior_rig}
\href{https://github.com/julianarhee/morph-pov}

\href{https://github.com/julianrhee/retinotopy-mapper}
\href{https://github.com/julianarhee/acquisition-tools}


% Data numbers 

\begin{table}[h]
 \caption{Data Summary}
  \centering
   \begin{tabular}{lllll}
    \toprule
    Stimulus & Area & Rats & FOVs & Cells   \\
    \midrule
    Moving bar & V1  & 5 & 12 & 1277        \\
               & LM  & 6 & 14 & 530         \\
               & LI  & 4 & 9 & 502          \\
    \midrule
    Receptive Fields & V1  & 5 & 12 & 1277  \\
                     & LM  & 6 & 14 & 530   \\
                     & LI  & 4 & 9 & 502    \\
    \midrule
    Gratings & V1  & 4 & 5 & 23     \\
             & LM  & 5 & 5 & 30     \\
             & LI  & 3 & 6 & 16     \\
    \midrule
    Objects  & V1  & 5 & 8 & 1493   \\
             & LM  & 5 & 7 & 1035   \\
             & LI  & 4 & 7 & 890    \\
    \bottomrule
  \end{tabular}
  \label{tab:data_counts}
\end{table}

% \section{Related to Chapter 3}

% \subsection{Comparison of stimulus size for receptive field measurements}

% \subsection{Demonstration of spherical correction}

% \subsection{Comparison of receptive field and gratings tuning}



% \section{Related to Chapter 4}

% \subsection{Tuning similarity as a function of distance}

% \subsection{Classifier accuracy as a function of receptive overlap}

% \subsection{Classifier generalization, matched for receptive field size}
