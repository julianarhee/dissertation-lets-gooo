\begin{savequote}[75mm]
This is some random quote to start off the chapter.
\qauthor{Firstname lastname}
\end{savequote}

\chapter{Systems for optical imaging in awake rats}

\newthought{To be able to visualize} single neurons as a population in an awake animals is amazing. Large-scale recordings of hundreds and even thousands of neurons are becoming increasingly standard, both using electrophysiology \cite{NEUROPIXELS ETC} and optical imaging \cite{STRINGER}. 

Additionally, transfection of the genetically-encoded fluorescent calcium indicators GCaMP6f and GCaMP7f using an adeno-associated virus (AAV) vector results in robust delivery of the GCaMP construct, and expression of the indicator remains robust and stable over at least several months (REFREF). In combination, these features would allow one to measure the same neurons in awake, behaving animals across large numbers of stimuli and trials over the course of weeks and months. In contrast, conventional acute single-unit microelectrode recordings are limited by the time that a single cell can be isolated (usually only one or a few hours). Even the best chronic preparations face difficulties holding isolated cells over very long time periods, and it is difficult to be absolutely confident that the same cell is isolated across days, especially given that nearby cells are often thought to have similar response properties.

\section{A tandem-lens macroscope for mapping brain areas}
I made a tandem-lens epifluorescence microscope \cite{Ratzlaff1991}.
% Figure:  Widefield epi setup

\section{Optical identification of visual areas in lateral cortex}
% Figure: Stimulus protocol + example maps + responses
% Figure:  Area segmentation and boundaries
Retintopy stuff.

\section{A tiltable 2-photon microscope}
% Figure: 2p schematic
% Figure:  2p retino/neuropil

We created a custom, 2-photon excitation microscope especially designed for recording from all visual areas in an awake behaving rat. A key feature of this microscope is its ability to pivot about the focal point of the microscope objective - the microscope can be tilted to any orientation required to access lateral cortical areas while keeping the animal in a natural, unrotated position. Meanwhile, the body of the microscope was constructed so as to allow ample space for the animal's body while maintaining over 180º of unobstructed viewing angle for a head-fixed animal.

We designed a kinematic implant for head-fixation that allows precise re-positioning of animals in the apparatus from day to day. In combination with blood vessel landmarks and the unique layout of fluorescent, GCaMP-expressing cell bodies in a given field-of-view in 3-dimensional space, this precise re-fixation allows us to rapidly relocate the same cells across sessions (\fig{REF}).  

\section{Simultaneous acquisition of high-resolution behavior}
% Figure: FACE TRACKING

The characterization of visual cortex also depends critically on knowing and controlling where stimuli fall on the retina.  To track eye movements, we developed a novel self-calibrating eye tracking technology capable of accurately measuring rodent eye movements in spite of the fact that traditional animal-loop calibration methods cannot be used in rodents \cite{Zoccolan:2010aa}.

