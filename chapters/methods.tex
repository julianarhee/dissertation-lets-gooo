\chapter{Methods}
\label{methods}

% Materials and methods
% ----------------------------------------------------
\section{Animals}
All experimental procedures were reviewed and approved by the Harvard Institutional Animal Care and Use Committee. Experiments were performed at Harvard University. Animals used in this study were female Long-Evans rats, 3 months or older, weighing 250-350g (Charles River Laboratories). Rats were housed on a ventilated rack under a 12 hour light:dark cycle with food and water ad libitum, except when water-restricted for behavior training. 

% ----------------------------------------------------
% Surgery
% ----------------------------------------------------
\section{Surgical Procedures}
\subsection{Viral injections}
Intracortical injections were made at multiple sites ($\sim$5-15 sites per cranial window, spaced 0.5-1mm apart) using a microinjector (NanoFil, World Precision Instruments) fit with a 36G beveled needle (NF36BV-2, WPI). A total of $\sim$500-750nl was injected per site at a constant rate of 10-25nl/min at a depth of 750$um$ below the surface. A high-titre solution of viral vector (AAV9-syn-jGCAMP7f-WPRE, Addgene) was diluted to a final ratio of 2:1 with a 20\% mannitol solution (Sigma-Aldrich) to promote diffusion. Trace amounts of Fast-Green (Sigma-Aldrich) were added for visual confirmation of injected solution in the brain.

\subsection{Headplate implantation}
Aseptic surgical technique was followed during all survival surgeries. A headplate and cranial window were implanted in the same surgery as viral injections using methods modified from mouse cranial window procedures \cite{Goldey2014}. Rats were administered dexamethasone (2 mg/kg) $\sim$3 hours prior to surgery in order to reduce brain swelling. Rats were anesthetized using isofluorane in 100\% O2 (induction, 3-5\%; maintenance, 1.5-2\%), and placed in a stereotaxic apparatus (Knopf Instruments, Angle Two, Leica). Eyes were protected from drying out ophthalmic ointment (Puralube), and then covered with surgical drape to protect from direct light. Heart rate, breathing rate, oxygen saturation, and body temperature were measured with a pulse oximeter and commercially available software (PulseOx, Mouseox). Body temperature was maintained at 38$\circ$C with a feedback-controlled heating pad. 

The top of the head was shaved above the incision site, followed by an application of Nair to clear the site of any hair prior to incision. The exposed scalp was cleaned with saline (), then wiped with three rounds of Povidone-Iodine swabs (Medline). A small lidocaine block (<0.5 cc) was administered along the incision site, which spanned from just behind the ears to the back of the head. After the incision, the skull surface was thoroughly cleaned with hydrogen peroxide (Swan) and a mixture of citric acid (10\%) and ferric chloride (3\%) (Parkell #S393). A series of small indentations were placed using a small drill all across the cleaned skull to increase surface area and texturize the skull in preparation for adhesives. 

Prior to head plate attachment, the center of the craniotomy was marked at -7.0 to -8.5 mm AP, 4.5 to 6.5 mm ML, depending on the areas being targeted for each animal. The implant procedure did not require any bone screws or additional supplements to keep the implant stable across months. A custom titanium head plate was attached to the skull over the right hemisphere using dental glue (C&B Metabond, Parkell S380). The head plate was placed at 40 degrees relative to Bregma, which matched the orientation of the imaging plane and captured most of the targeted areas of visual cortex. 

\subsection{Cranial window}
After the head plate was securely glued to the skull surface, the wound margin was closed up, while leaving the circular region surrounding the craniotomy site exposed. A 4-5mm diameter craniotomy was performed at the marked site by careful thinning of the bone with a dental drill within the circular area (Aseptico). Care was taken throughout the drilling process to keep the thinned region within the circular boundaries using a pair of surgical calipers (Fine Surgical Tools). Skull thinning was complete once the entire circular region was semi-transparent and blood vessels were clearly visible through the thinned skull. Once the skull was thinned down, the region was kept immersed in sterile saline for the remainder of the surgery. The remaining thinned bone was removed with laminectomy forceps (Fine Science Tools). The dura was cut open using a beveled 36G needle tip that was bent such that the fine tip curved upward. This was effective for gently lifting up the dura enough to create a small incision point, without risking pressure or puncture on the cortical surface beneath the dura. Flaps of dura were then peeled back with fine forceps or spring scissors to expose the brain surface, and tucked away around the edges of the craniotomy. Intracortical injections were performed after the duratomy while the entire area was submerged in sterile saline (see Viral Injections).

A window composed of stacked glass coverslips (four to five 4mm, plus one 5mm, Warner Instruments) bound with optical adhesive (Norland #71) was then placed over the brain surface. The remaining saline was partially absorbed out with sterile eye-spears, and the craniotomy was sealed with cyanoacrylate glue (Vetbond, 3M) over a thin layer of sterile saline. Post-operative animals were administered buprenorphine (0.01-0.05 mg/kg) and carpofen (5 mg/kg) daily for 5-7 days following the surgery.



% ----------------------------------------------------
% Area identification
% ----------------------------------------------------
\section{Wide-field mapping}
\subsection{Animal preparation}
About 20 minutes prior to the mapping session, animals were anesthetized with isofluorane (5\% induction, 1-1.5\% maintenance) and administered a subcutaneous dose of cholorprothixene (2 mg/kg, Sigma-Aldrich). During the mapping session, animals were kept anesthetized with minimal levels of isofluorane (0.5-1\%). Anesthesia levels were maintained by testing the paw-pinch reflex and monitoring the breathing rate. The left eye facing the monitor was checked between trials to ensure it remained open and stable. Multi-site viral injections allowed for greater coverage of exposed cortex, but resulted in patchy expression levels in a subset of animals. Animals with ambiguous retinotopic maps were excluded from further study.

\subsection{Tandem-lens macroscope}
Widefield mapping was done with a tiltable, tandem-lens macroscope\cite{Ratzlaff1991, Kalatsky2003}, composed of a USB 3.0 CCD camera (MantaG033-B, Allied Vision) and 2 Nikon lenses (Nikon, 105-mm and 55-mm). Images were acquired at 25 Hz with 3x3 pixel binning (256x492 pixel ½-type sensor). Epifluorescence illumination was achieved with a 470nm LED (Thorlabs) that was filtered and reflected through a filter cube that housed an excitation filter, dichroic mirror, and emission filter (Thorlabs). Green fluorescence or reflected light was collected and passed through the filter cube then focused on a CCD detector. For epifluorescence illumination, we used a 470 nm LED filtered and reflected by a long-pass dichroic mirror, and emitted fluorescence was filtered and captured at an imaging rate of 25Hz using custom Python scripts.

% Compare w. Wekselblatt et al. 2016:  Camera lenses allow a relatively high numerical aperture (NA) for light collection, which can also be adjusted easily using the f-stop setting to restrict the NA. This permits a flexible trade-off between sensitivity and depth of field, especially as increased depth of field is useful, given the curvature of the cortical surface. Imaging was generally performed at an f-stop of 5.6. The ratio of the focal lengths of the two lenses determines image magnification. To map 1 cm of cortex across the 2-cm detector (6.5 μm pixels), we chose 50 mm and 105 mm lenses, yielding magnification of 2.1× and 3.1 μm specimen pixels. In practice, we find an effective spatial resolution of ∼25 μm, based on the highest spatial frequencies present in nonbinned images of vascular structure. Binning across spatially oversampled pixels can reduce shot noise by allowing more total photons to be detected with increased illumination or NA. This is a standard practice in intrinsic signal imaging (Kalatsky and Stryker 2003) and is generally applicable at high light levels, where readout noise is negligible compared with photon count noise.

\subsection{Visual stimulation}
Visual stimuli were presented using custom Python scripts on a 72 inch LCD monitor (LG). The monitor was centered in front of the left eye, spanning 177 degrees of visual field along azimuth, 67 degrees along elevation. The mapping protocol consisted of a periodic, moving bar stimulus \cite{Kalatsky2003, Marshel2011} presented to the (left) eye contralateral to the cranial window. The bar subtended 5 degrees of visual angle, and was either presented as a white bar drifting over a black background or an apertured bar containing one of 32 possible natural scene images drifting over a gray background. The bar was presented at 0.13 Hz along the azimuth and elevation axes, for a total of 2 (downward, rightward) or 4 (downward, rightward, leftward, upward) conditions. A total of 4-5 repetitions of 10 cycles each were acquired for each direction. To preserve the speed of the bar between azimuth and elevation conditions, the bar traversed the full extent of the monitor's width centered along the monitor's vertical extend (bar started and ended off screen). 

\subsection{Area segementation}
Raw fluorescence signals were corrected for slow drift by removing the rolling average of each pixel’s time course. The width of the rolling window was set to 2.5 times the length of the stimulation period. For each pixel, the time courses for each trial (10 cycles of the stimulus) were aligned and averaged for each condition (1 of 4 possible directions). We then performed a Fourier spectral analysis on the averaged time series for each pixel to get a magnitude and phase value for each pixel at each frequency. The strength of the response to the stimulus was calculated as the ratio of the Fourier magnitude at the frequency of stimulation to the sum of the magnitudes at all other frequencies\cite{Kalatsky2003, Marshel2011, Juavinet2017}.

Retinotopic maps were created by taking the phase values for all pixels in the image and converting them to Cartesian coordinates that matched the linear position of the bar on the monitor to the phase of the stimulus cycle that corresponded to that position. All maps were smoothed with a Gaussian window (FWHM=$50um$) and masked by applying a threshold to the magnitude ratio.  

% ----------------------------------------------------
% 2p imaging
% ----------------------------------------------------
\section{Two-photon calcium imaging}
\subsection{Image acquisition}
Data was collected using a custom-built galvo-resonant scanner two-photon microscope (20 kHz; Cambridge Technologies) and a 0.8 WD/16x objective lens (CF 175, Nikon Technologies). 920 nm excitation provided by a Mai Tai DeepSee laser (Spectra-Physics) was used for both channels. The microscope body was tilted at an angle of 40 degrees to match the plane of the cranial window. Single plane images were collected at a rate of 44.65 Hz (512x512 pixels; 500μm x 500 μm FOV) using Scanimage (Vidrio Technologies). For each session, the FOV was placed
150-300 um below the brain surface within a patch of visual cortex representing the part of the
visual field overlapping with the screen.

\subsection{Data processing}
Motion correction, cell body and neuropil identification, and trace extraction were done with a custom pipeline of Matlab and Python scripts. Motion correction used rigid transformations within each FOV using custom Matlab code (Chris Harvey, Harvard Medical School). For ROI selection, an activity map was created by taking the standard deviation across motion-corrected frames within a movie file for each block of trials, and then taking the maximum projection across all blocks. ROIs were selected manually with a circular mask using a custom Matlab GUI. Neuropil correction was done by taking an annulus around the soma, and subtracting the neuropil trace multiplied by a correction factor to produce the neuripil-corrected soma trace. The correction factor was set to 0.7. For a subset of data, both Suite2p and CaImAn were tested.

Average fractional change in fluorescence during the 1-second stimulus presentation compared to a 1-second baseline period immediately prior to stimulus onset.

\subsection{Area identification and validation}
We targeted a given two-photon FOV by coregistering blood vessel images to wide-field retinotopic maps, which allowed coarse-grained targeting of two-photon sessions. All two-photon imaging sessions began with an anatomical run, in which we acquired a surface level z-stack for fiduciary markers to be used in map registration. For blood vessel images, rats were given subcutaneous injections of SR101 (Sigma-Aldrich) for visualization in the red channel. If blood vessel images were unavailable, the green channel was used for FOV alignment. 

Two-photon blood vessel images were matched to wide-field maps offline. We selected matching points between the two views based on uniquely identifiable blood vessels present in both views, then used these points to identify a transformation matrix to warp one into the other. Assignment of two-photon FOVs were validated based on retinotopic maps measured with the same paradigm used for wide-field maps of azimuth and elevation. We generated sign maps from the retinotopic maps with a series of morphological filters\cite{Marshel2011, Garrett2014, Zhuang2017}, which were then used to identify patches representing putative visual areas. Only two-photon FOVs with matches to wide-field maps and unambiguous sign reversals were included for subsequent analyses. 

To validate area assignment for FOVs initially selected based on wide-field maps, we repeated the cycling bar mapping procedure in all two-photon experiments. A white bar subtending 2-5 degrees of visual angle cycled across a black screen at a stimulation frequency of 0.24Hz or 0.13Hz. Each of four cardinal directions were tested for more careful retinotopic estimations (downward, upward, leftward, rightward). Each trial consisted of of 12 cycles of the stimulus, and 4-6 trials were presented for each of the four conditions, totalling 16-24 trials total. Blank trials were also included to measure baseline fluctuations in spontaneous activity.  

% \subsubsection{Estimation of retinotopic preferences in cell bodies and neuropil}

% \subsubsection{Estimation of cells with significant visual responses}
% stuff

\subsubsection{Preferred retinotopic location}
Retinotopic responses of neuropil surround cell bodies
The center of each neuropil ring was assigned the value corresponding to the preferred retinotopic location of that neuropil ring, a disk of 10um radius was dilated from the center. Overlapping disks were averaged for preferred retinotopy estimates of neuropil. The final pixelwise estimates were the result of spatially smoothing with a Gaussian filter. 
% (Gauss sigma=7 → microns). 

% Rate of retinotopic change -- methods from Liang et al., Cell 2018. - “rate of retinotopic change” for smoothed retinotopic map.

\subsubsection{Estimation of receptive fields}
To estimate receptive field size and positions, we adapted a standard retinotopic mapping protocol~\cite{Marques2018}. The screen was tiled into either 21x11 positions of 5 degrees or 11x6 positions of 10 degrees. Drifting square-wave gratings (spatial frequency 0.25 Hz, speed 10 deg/s) were presented in a single tile at each position, and pseudo-randomly switched between the 4 cardinal directions every 125 ms during the 500 ms stimulus presentation window. Each position was repeated 10 to 20 times. 

% Receptive Fields.
% Retinotopic responses of neuropil surrounding RGC axonal boutons Retinotopic tuning curve fitting was also implemented for the neuropil rings surrounding each bouton, to estimate the local retinotopic preference in the field of view. Each pixel in the field of view was attributed a preferred retinotopic location by first assigning the center of each neuropil ring with a value corresponding to the preferred retinotopic location of that neuropil ring, then dilating by a disk of 10 mm radius from each neuropil center respectively and averaging the preferred retinotopy across overlapping disks. The final pixel-wise estimates of retinotopic preference were obtained by spatial smoothing using an isotropic two-dimensional Gaussian filter with a standard deviation of 3 mm. The rate of change of retinotopy along the field of view (which was tangential to the surface of dLGN) was measured along the axis
% for which the retinotopic map changed the fastest. To compute this spatial axis, we first calculated the two-dimensional pixel-wise gradient: VRetðx;yÞ = ðvRetðx;yÞ=vxÞbi + ðvRetðx;yÞ=vyÞbj. The spatial axis was defined as the normalized mean gradient vector across
% pixels, VRet=kVRet k . The smoothed retinotopic mapwas then projected onto the normalized mean gradient vector (i.e., onto the unit vector along the direction of maximal change in retinotopic preference). For each pixel, we derived its projected location along this new spatial axis as: ~x = ðx; yÞ,VRet=kVRet k . The relationship between the preferred retinotopic location and ~x was modeled accord- ing to a linear function with offset: Retfitð~xÞ = a~x + b. The fitted parameter a (units: deg/mm) indicated the progression rate of the smoothed retinotopic map. We computed the normalized mean gradient axis and the scale factor, a, both for maps of azimuth and


% Fine-scale retinotopic scatter Fine-scale retinotopic scatter (‘deg scatter’, Figure S1F) was estimated as the absolute retinotopic deviation, Sret (units: degrees of
% ?
% ?
% visual space): Sret =
% ?Retprefð~xBÞ? ða~xB + bÞ
% ? . Here,Retprefð~xBÞ denoted the preferred retinotopic location of a given bouton, while
% ða~xB +bÞ gave the predicted receptive field center based on the neuropil estimate, according to the projected location ~xB along the mean gradient axis. We also calculated the absolute deviation in spatial distance from the fitted spatial progression in the field, Sspa
% ?
% ?
% (‘distance scatter’, Figure S1F; units: mm): Sspa =
% ?~xB ?ðRetprefð~xBÞ? bÞ=a
% ? . This value is equivalent to the distance that a bouton
% would need to be moved along azimuth or elevation in order to obtain a perfectly smooth map.
% Bouton


\section{Quantification and statistical analysis}
For all pairwise tests, Wilcoxon signed-rank tests were used, unless otherwise specified. Significance values were set to p<0.05 or p<0.01. For all comparisons between visual areas, Mann-Whitney U-tests, with significance values corrected for multiple comparisons.

\subsection{Visual Stimuli}
\textit{Objects}. Visual objects were renderings of three-dimensional models built using a ray tracer package POV-Ray (http://www.povray.org/). Each object was defined as a particular configuration and blend of spheres. The particular objects selected for the anchors were modeled to replicate the stimuli used in a previous study\cite{Zoccolan2009}. Figure\ref{fig:basic_training} shows the "default" object views used during phase 1 of training. Objects were rendered with the same light source location and matched to have approximately equal height, width, and area, as defined by the a bounding box surround each object rendering. Object transformations (\textit{e.g.}, in-depth rotation) were generated using custom Python wrappers and the POV-Ray API). Morph stimuli were generated by gradually adjusting the relative proportions of each object, by parametrically shifting the spheres defining one object into the spheres defining the other. We used the Euclidean distance in pixel space to quantify the difference between each neighboring pair of images. 

For two-photon imaging experiments, a subset of these morph stimuli were used: Each morph (7 intermediate morphs, plus 2 anchors) was presented at 5 different sizes (10-50 degrees, in 10 degree steps). For each size, mean luminance was measured with a photometer to determine the levels needed to create full-screen controls matched for each size and object. This resulted in 50 unique object conditions, presented across ~30 repetitions each.

% ----------------------------------------------------------
% GRATINGS
% ----------------------------------------------------------
\section{Estimation of tuning preferences}
To measure visual feature tuning, we presented drifting gratings at 8 directions (0 to 315 degrees, steps of 45 degrees). Gratings were presented at either full screen or apertured, with the aperture size determined by the average receptive field size of the population recorded in previous  mapping sessions.  All directions were also presented at two spatial frequencies (0.5 and 0.1 Hz) to target the low and high end of known visual acuity levels~\cite{stuff}, and at two speeds (10 deg/s and 20 deg/s). This set of stimulus configurations resulted in 64 unique grating stimuli, which were repeated $\sim$20 times in pseuedo-random order.

\subsubsection{Direction tuning}
For each cell and each stimulus condition, we determined the evoked response to be significantly different from noise if the amplitude of dF/F during the stimulus period exceeded 2.5 standard deviations above or below the mean baseline activity (computed using 0.5s or 1s prior to stimulus onset) for at least 10 out of the 44 time points (44.65 Hz frame rate * (0.5 s stimulus presentation + 0.5 s after stimulus offset).

To determine if a given cell showed a significant response at a particular stimulus configuration (a unique combination of size, speed, spatial frequency), we required that at least 2 out of the 8 directions at this given stimulus configuration evoked responses according to the above criteria. 

Axis selectivity, direction selectivity, and preferred theta were calculated as described in Liang et al., 2018\cite{Liang2018a}. For bootstrap analyses, we performed 1000 iterations of 20 samples each, to match the measured sample of gratings trials. 

% Include axis-tuning and axis selective ind

% \textbf{\textit{Axis and direction selectivity}}
% Include axis-tuning and axis selective index (ASI) and direction selective index (DSI) calculations and equations. Bootstrap analysis, goodness-of-fit.

% \textbf{\textit{Preferred direction of motion and preferred axis of motion}}
% Bootstrap analysis, goodness-of-fit.

% \textbf{\textit{Preferred spatial frequency}}
% Bootstrap analysis, goodness-of-fit.

%%%% 


% Data acquisition
% Visual stimuli, synced to two-photon acquisition, ScanImage vX.X (REFREF). 
% Synced also to face-tracker camera.
% Image preprocessing
% stuff
% Cell mask identification
% stuff
% Time course extraction and correction
% Stuff


\subsubsection{Selectivity and tolerance metrics}
Neuronal selectivity to morphs was quantified by a morph tuning index\cite{Zoccolan2007}, MT=n-Ri/Rmax/n-1, where Ri is a neuron’s response to the ith morph, Rmaxis the maximum response amongst the morphs, and nis the number morphs. As a measure of response sparseness, MT ranges from 0 (no shape selectivity) to 1 (maximally shape selective). Size tolerance was quantified by normalizing size tuning curves to their maximum values, then averaging those resulting values that were < 1, that is, ST=Rtest/maxRtest, where Rtest is the mean response to a given test size of a neuron’s most preferred object, and . denotes the average across tested sizes where Rtest<maxRtest. 


\section{Population readout}
\subsubsection{Discriminability}
To quantify discriminability (Figure 5b), we trained linear classifiers (support vector machines) to discriminate the two original objects from the neural activity in each area. The linear-readout scheme is important in that it is a simple, biologically plausible processing step that amounts to a thresholded sum taken over weighted synapses. This classifier approach does not provide a measure for the total information present in the population, but rather estimates the lower bound on the information explicitly accessible to the population to support the visual task\cite{Hung2005, Rust2010}.

Linear support vector machines (SVMs) were trained to discriminate object A from object B from neural responses. Each presentation of an image produced a population response vector x of size Nx1, such that repeated presentations form a cluster of points in N- dimensional space (see \ref{fig:neural_generalization}). The linear readout amounts to finding a linear hyperplane that best separates the response clouds corresponding to each image from those corresponding to all other images. Specifically, the linear readout amounts to: f(x)=wTx + b, where w is a Nx1 vector of the linear weights applied to each of N neurons (defines the orientation of the hyperplane), and b is a scalar that offsets the hyperplane from the origin. To determine the population’s choice about which image was presented, a response vector x (population response to one image) was applied to the classifier, and negative values of f(x) indicate object A and positive values indicate object B. Performance was defined as the proportion of correct answers when asked to classify each image with a held-out test set never included in training. 

The hyperplane and threshold for each classifier was determined by a support vector machine (SVM) using the scikit-learn machine learning library (LinearSVC\cite{Pedregosa2011}). The data were split into train and test sets (20\%) after balancing numbers of samples per condition. We used a 5-fold cross-validation procedure on the train set to fit and evaluate each model:  of the trials partitioned for training, models were optimized where the best C={10-3, 10-2, 10-1, 1, 10, 102, 103}that yielded the highest accuracy was chosen. Performance was then assessed with the held-out test set. 

\subsubsection{Linear separability and generalization}
To test linear separability (Figure 5c), 80\% of the trials corresponding to object A and object B for each size were simultaneously combined to train and evaluate the models, while the remaining 20\% of trials was used to measure classifier performance. To test generalization, two regimes were used. In the first, “Train one, test one” (Figure 5d-e), each classifier was trained to classify object A and B at one of the 5 sizes, then tested on each of the remaining 4 untrained sizes. Each training set (for each size) included 80\% of the trials for a given size, while the test sets could contain either the remaining 20\% of trials of the same size (test accuracy on “trained” conditions) or 100\% of the trials at one of the other sizes (test accuracy on “novel” conditions). 
In the second regime, “Train a subset, test one” (Figure 5f),  each classifier was trained with trials of  4 of the 5 sizes together, then tested with the remaining, untrained size. Each training set was composed of 80\% of trials at each of the 4 train sizes. The remaining 20\% of trials for each of the 4 train sizes combined to form the held-out test set (test accuracy on “trained” conditions), while 100\% of the trials of the remaining 5th size formed the other type of test (test accuracy on “novel” conditions). All combinations of 4 of 5 sizes were used to train different classifiers and test generalization performance on samples of the remaining 5th size.

\subsubsection{Population sampling}
In analyses in which a given metric, e.g,. classifier accuracy, is presented as a function of the number of neurons in a pseudo-population, we applied a resampling procedure to measure the variability that can be attributed to the particular subpopulation of neurons or particular subset of trials used for training versus testing. On each iteration, we sampled a new subpopulation of neurons that were randomly selected (without replacement) from all cells aggregated across imaging sites and animals, for a given visual area, and trials were randomly assigned for training and testing (without replacement). Error bars were calculated as the SD of classifier performance across 100 iterations. Chance performance was computed by randomly assigning objects or images associated with each response vector and repeating the classification analysis.

