\chapter{Introduction}
\label{introduction}

Our brain's visual system is an amazingly complex information processing system that transforms a flurry of photons arriving at the retina into a coherent understanding of objects and surfaces in the environment~\cite{DiCarlo2007, Cox2014}. The hierarchical organization of the visual system is thought to play a key role in its ability to extract and learn about latent structure from sensory inputs. While much progress has been made in understanding the earliest stages of visual processing, our understanding of how higher visual cortical areas integrate low-level image features into representations of objects remains unclear. Understanding how the brain processes sensory information is a fundamental pillar in understanding how the brain works. 

The ability to recognize objects despite the enormous variation with which they can appear, \textit{e.g.}, due to viewing angle, scale, or lighting, is called invariant object recognition. The ease which we can recognize objects belies the computational complexity of this problem (EXAMPLE FIG REFREF?). We know that visual cortex successively processes input in a hierarchical cascade: lower-level areas of visual cortex tend to represent local features of the retinal image (\textit{e.g.}, oriented edges in local image patches), while subsequent higher-level extrastriate areas represent increasingly more abstract properties of the external world (\textit{e.g.}, object identity). However, the computations carried out by visual cortical populations to enable this increasing complexity remains a mystery.

% What do we know about visual areas (properties)
In primates, visual cortex is traditionally divided into two parallel pathways: the “ventral” pathway processes visual shape or form, while the “dorsal” pathway processes visual motion and spatial relationship~\cite{REFREF}. Visual object recognition in primates is thought to emerge through sequential processing stages across cortical areas in the ventral visual pathway~\cite{Rust2010SelectivityIT, DiCarlo2007, DiCarlo2012, Chen2014}.  EXAMPLE FIGURE HERE?

From early to late stages of the hierarchy, single neuron properties vary systematically:  receptive field sizes get larger, selectivity for complex shapes increases~\cite{Desimone1984, Logothetis1996}, and tolerance to identity-preserving transformations~\cite{REFREF}. 

% What's holding us back!
 While much work has been done studying the responses of single neurons on the visual pathway, we lack a complete understanding of how populations of neurons work together to represent the external world. Our understanding of the nature of computations that take place in visual cortical populations have been partially limited by the tools at our disposal to probe neuronal populations.   
 
 Investigators have long sampled from neuronal populations in the ventral visual pathway of monkeys using single microelectrodes or electrode arrays, but these techniques have several fundamental limitations. First, they are only able to sample from a comparatively small number of neurons at time, generally requiring the assembly of serially-sampled ``pseudo-populations'' in order to explore population coding. Second, electrophysiological recording techniques typically do not allow the same populations of neurons to be studied over long periods of time. In some cases, a small number of neurons can apparently be isolated for days in a chronic preparation, however, it is usually impossible to be certain that these cells are the same across days.

However, primates are a difficult animal system in which to use existing tools for probing neuronal populations, such as \textit{in vivo} 2-photon calcium imaging \cite{Ohki2005}: monkey cortex is 2.5mm thick on average \cite{Koo:2012aa}, making it difficult to image much beyond layer II, due to optical scattering limits. In addition, monkeys are capable of significant head movement even when head-fixed, and even at rest their brains move significantly due to pulsatile motion, which complicates imaging. At the same time, the complexity of the monkey ventral visual pathway, albeit more similar to the human visual system, is more challenging to understand, relative to simpler systems. As a result of all of these factors, a growing number of investigators have turned to simpler alternative models to explore a variety of ``high-level'' sensory and cognitive functions in less complex systems (e.g. \cite{kepecs2008neural, zeeb2009serotonergic}).

In contrast, rodents offer excellent experimental accessibility: for example, greater genetic manipulability, the option of studying large numbers of individual animals, and a smaller cortex in which 2-photon imaging techniques can access deeper layers of cortex. 

To date, the majority of studies of higher-level visual processing have been done in non-human primates, largely because they are similar to humans, and because it has been erroneously assumed that simpler model systems lack sophisticated visual systems. Increasing evidence suggests rodents possess sophisticated visual machinery that would make them a tractable model for studying multi-level visual processing.

Rats, in particular, can perform invariant object recognition tasks~\cite{Zoccolan2009, Tafazoli2012Transformation-TolerantPriming, cite}, and their visual cortex, namely, areas V1, LM, LI, and LL, exhibits properties similar to that of primates~\cite{Tafazoli2017, Vermaercke2014, Matteucci2019b, cite}. {NAME SOME PROPERTIES}

Additionally, rodents offer excellent experimental accessibility: for example, greater genetic manipulability, the option of studying large numbers of individual animals, and a smaller cortex in which 2-photon imaging techniques can access deeper layers of cortex. Rodents therefore are an attractive model system for the study of visual processing.

Critically, many groups \cite{EVERYONE} have found many of the key hallmarks of the primate ventral visual pathways in the lateral visual cortex of rats: visual object selectivity that increases from beyond to extrastriate areas, in combination with growing receptive field sizes and signs of tolerance to identity-preserving transformations, such as variation in position or scale. These findings suggest that rodent extrastriate areas can serve as a valuable complement to monkey visual models, but many aspects of the rodent visual pathway remain to be characterized in detail. The aim of the present study was to expand on existing characterizations of response properties in rat extrastriate visual cortex using calcium imaging.


Here, we take advantage of optical methods, which allow simultaneous access to multiple brain regions with single-cell resolution, and present their first application to an animal model of invariant object recognition. We describe the macro- and micro-scale organization of V1, LM, and LI in the context of a range of stimulus classes. Specifically, we characterize feature tuning and object tuning in rat visual cortex to examine how neural circuits encode feature variations that either preserve object identity or alter object identity. 


The present work aims to fill gaps in our current understanding of the rodent visual pathway that will further enhance the rodent as an animal model that leverages greater experimental accessibility to more deeply probe fundamental questions about how visual information is organized and processed in the mammalian brain. Taken together, this work offers the potential for a new, powerful, complementary animal model for the study high-level vision.



%%%% 

% Vision is critical for guiding behavior in species throughout the animal kingdom. Though nonhuman primates have been the classic model for vision, alternative systems offer a way to reveal general principles of visual processing as well as species-specific adaptations. Advances in imaging techniques and genetic tools have made genetically accessible animals, like mice, valuable systems for understanding neural circuits of vision. In contrast, although rats have long been studied for visually-guided behaviors, large-scale cellular resolution access to their visual cortex has proved to be far more challenging. Here, we aim to bridge the gap between neural circuit access and rich behavioral capacities in the rat. We characterize the functional organization of a subset of visual areas for the first time with cellular resolution imaging in awake, head-fixed animals. 