\chapter{Introduction}
\label{introduction}

% hierarchy, vision, Hubel & Wiesel, 1962, J Physiology
Our brain's visual system is an amazingly complex information processing system that transforms a stream of photons arriving at the retina into a coherent understanding of objects and surfaces in the environment~\cite{DiCarlo2007, Cox2014}. The hierarchical organization of the visual system is thought to play a key role in its ability to extract and learn about latent structure from sensory inputs. While much progress has been made in understanding the earliest stages of visual processing, our understanding of how higher visual cortical areas integrate low-level image features into representations of objects remains unclear. Understanding how the brain processes sensory information is a fundamental pillar in understanding how the brain works. 

The ability to recognize an object across the multitude of ways it can appear, \textit{e.g.}, due to viewing angle, scale, or lighting, is called invariant object recognition, or transformation-tolerant recognition. Any particular encounter with an object can cast a dramatically different image onto the retina. Despite this tremendous variation, we are able to rapidly detect, discriminate, and recognize thousands of distinct object classes with apparently little effort. The ease which we can recognize objects belies the computational complexity of this problem. We know that visual cortex successively processes input in a hierarchical cascade: lower-level areas of visual cortex tend to represent local features of the retinal image, \textit{e.g.}, oriented edges in local image patches, while subsequent higher-level extrastriate areas represent increasingly more abstract properties of the external world, \textit{e.g.}, object shape and identity. However, the computations carried out by visual cortical populations to enable this increasing complexity remains a mystery.

The mammalian visual system is the most well-studied sensory modality. Over the past 50 years\cite{Hubel1962}, the idea of a hierarchically organized system has framed our understanding of how complex behavior arises and has inspired powerful multi-layered computational networks that are capable of impressive feats\cite{Riesenhuber1999, Krizhevsky2012}. Understanding how the brain builds up visual object representations has the potential to inform not only how we perceive the things in our world, but more broadly, how a wild disarray of sensory information can be transformed into complex representations that can guide meaningful behavior.

% ----------------------------------------------------
% Primate visual system
% ----------------------------------------------------
% What do we know about visual areas (properties)
\section{The primate visual system}
In primates, visual cortex is traditionally divided into two parallel pathways: the ``ventral'' pathway runs along the ventral side of the occipital and temporal lobes, and is associated with visual shape processing, while the  ``dorsal'' pathway runs along the dorsal side of the occipital and parietal lobes, and is associated with motion processing and spatial relationships~\cite{Mishkin1982CONTRIBUTIONMONKEYS, Ungerleider1994WhatBrain, Felleman1991}. Early evidence for separate processing streams comes from studies in which lesions in different parts of these cortical paths produced contrasting effect in monkeys. For example, lesions of the inferior temporal (IT) cortex cause severe deficits in visual discrimination tasks, but have no effect on visuospatial tasks, such as visually-guided reaching or relative judgements of visual distances. In contrast, parietal cortex lesions, along the dorsal pathway, do not affect performance on visual discrimination, but severely disrupt visuospatial behaviors. As such, the dorsal path is referred to as the ``where'' pathway (concerned with \textit{where} an object is), and the ventral path is often called the ``what'' pathway (concerned with \textit{what} an object is)\cite{Ungerleider1994WhatBrain}. These pathways have also been framed in terms of a perception/action dichotomy --- that is, recognizing an object versus reaching toward it\cite{Goodale1992}.

While the two-stream framework is certainly an oversimplification, what we do know is that the primate visual cortex extracts and transforms information from one area to the next. The process begins at the retina, where photoreceptors transduce photons into neural activity, which then travels through a series of brain regions, from the lateral geniculate nucleus (LGN) and on to primary visual cortex (the first cortical area, also known as V1, or striate cortex), and beyond. Each step in this process extracts increasingly refined features, such as neurons in V1 that are selectively responsive to edges of particular orientations, and finally culminate in a group of visual areas beyond V1, collectively called "extrastriate" cortex --- here, neurons respond selectively to higher-order visual features, such as objects, faces, or motion\cite{Orban2008, Dicarlo2012}. 

Visual object recognition in primates is thought to emerge through these sequential processing stages across the ventral visual pathway~\cite{Rust2010SelectivityIT, DiCarlo2007, DiCarlo2012}. From early to late stages of the hierarchy, single neuron properties vary systematically. In V1, receptive fields are smaller and cells are optimally stimulated by oriented edges or gratings\cite{Hubel1968}). In subsequent stages, receptive field sizes get larger, while selectivity for complex shapes increases~\cite{Desimone1984, Logothetis1996}. At the same time, neurons in later, higher-level areas exhibit tolerance to identity-preserving transformations: in these highest stages of the ventral pathway, inferotemporal or IT cortex, cells selectively respond to a given object regardless of changes in size, location, particular appearance on the retina~\cite{Tanaka1996, DiCarlo2012}. Some studies even report neural activity highly specific for behaviorally-relevant categories, such as faces\cite{Kanwisher1997, Tsao2006} and scenes\cite{Epstein1998}.

% ----------------------------------------------------
% Methods for population recordings
% ----------------------------------------------------
% What's holding us back!
\section{Methods for population recordings}
While much work on the visual system has been done studying the responses of single neurons, we lack a complete understanding of how populations of neurons work together to represent the external world. Our understanding of the nature of computations that take place in visual cortical populations has been partially limited by the tools at our disposal to probe neuronal populations. Decades of research in the primate visual system have relied on investigators sampling from neuronal populations in the ventral visual pathway of monkeys using single microelectrodes or electrode arrays, but these techniques have several fundamental limitations. First, they are only able to sample from a comparatively small number of neurons at time, generally requiring the assembly of serially-sampled ``pseudo-populations'' in order to explore population coding. Second, electrophysiological recording techniques typically do not allow the same populations of neurons to be studied over long periods of time. In some cases, a small number of neurons can apparently be isolated for days in a chronic preparation, however, it is usually impossible to be certain that these cells are the same across days. 
% Ephys:
% (Jun et al., 2017 (neuropixels); 
% Siegle et al., 2019 --Nature, 2021,. (hierarhcy, Nature)
% Stringer et al., 2019a -- spontaneous bebaviors, Science

% ca imaging: 
% (Sofroniew et al., 2016; -- meoscope, elife
% Stringer et al., 2019b --high-dim geometry, Nature
% Weisenburger et al., 2019). --- alipasha, Cell
A major goal of systems neuroscience is to explain how complex behaviors arise from the combined activity of many individual neurons. Thus, advances in population recordings, which allow the activity of many neurons to be captured at the same time, have been dramatically changed the nature of questions that researchers are able to ask. Extracellular electrophysiology and two-photon optical imaging are the most popular methods for recording neural populations, due to their single-cell resolution, and relative ease of use, and scalability. Within the last five years, large-scale recordings of hundreds and even thousands of neurons are becoming increasingly standard using electrophysiology \cite{Steinmetz2019, Siegle2021, Stringer2019SpontaneousActivity} and optical imaging \cite{Stringer2019High-dimensionalCortex, Weisenburger2019, Sofroniew2016}. With rapidly growing advancements in genetic tools, hardware design, and computational power, we are in an unprecedented time to be studying neural circuits and behavior. 

Today, one of the greatest advantages of optical imaging is cellular resolution access to large neural populations. To be able to visualize single neurons as a population in awake animals is immensely powerful, especially in concert with sophisticated manipulation techniques, such as multi-channel optogenetics and holographic light stimulation \cite{Gill2020, Chong2020 Packer2016, Yang2018}, and genetically-defined targeting of specific populations of cells. In species in which these tools are available, such as mice, we have been able to gain unprecedented access to genetically defined cell types and their role in circuit functions\cite{Luo2008, Luo2018, HubermanNiell2011}. In combination, these features allow one to measure and manipulate the same neurons in awake, behaving animals across large numbers of stimuli and trials over the course of weeks and months. 


However, primates are a difficult animal system in which to use existing tools for probing neuronal populations, such as \textit{in vivo} 2-photon calcium imaging \cite{Ohki2005}: monkey cortex is 2.5mm thick on average \cite{Koo2012}, making it difficult to image much beyond layer II, due to optical scattering limits. In addition, monkeys are capable of significant head movement even when head-fixed, and even at rest their brains move significantly due to pulsatile motion, which complicates imaging. At the same time, the complexity of the monkey ventral visual pathway, albeit more similar to the human visual system, is more challenging to understand, relative to simpler systems. As a result of all of these factors, a growing number of investigators have turned to simpler alternative models to explore a variety of ``high-level'' sensory and cognitive functions in less complex systems (e.g. \cite{Brunton2013, Miller2017TwoStep, Aronov2014, Glickfeld2017, HubermanNiell2017}).

% ----------------------------------------------------
% Rodent Models for Vision
% ----------------------------------------------------
\section{Rodent Models for Studying Vision}
Over the past ten years, rodents have emerged as a powerful system for studying visual circuits. Among rodents, this is particularly true for mice, as an important driving force has been the rapid development of a large array of genetic tools for analyzing connectivity and probing and controlling activity in neural circuits\cite{Luo2008, Luo2018}. Furthermore, rodents offer excellent experimental accessibility. They have a lissencephalic cortex, which means that their cortex is smooth and do not have the characteristic folds of gyri seen in primate cortex. This is particularly advantageous for optical imaging techniques, as more of the brain is accessible right at the surface. Second, rodents cost much less to keep in the laboratory and are easier to house in large numbers, which means access to larger sample sizes in a given experiment. With their smaller size comes a much smaller brain, relative to primates, which, theoretically, may facilitate understanding, as more of the brain can be recorded from per animal. On the other hand, mice and rats are have much lower visual acuity, which raises the possibility that their visual system is fundamentally different from the visual systems of primates. 

%Evidence that rats are good at stuff
To date, the majority of studies of higher-level visual processing have been done in non-human primates, largely because they are similar to humans, and because it has been erroneously assumed that simpler model systems lack sophisticated visual systems. However, increasing evidence suggests rodents possess rather sophisticated visual machinery that would make them a tractable model for studying multi-level visual processing. A number of studies have demonstrated that many basic properties of visual function, at least from the retina and up to V1, are present in rodent visual cortex\cite{HubermanNiell2011}. Within V1, several circuits underlying a range of cortical computations have been found, including orientation selectivity \cite{Ko2011, Lien2013TunedCircuits}, surround suppression\cite{Adesnik2012}, and gain control\cite{Atallah2012}. 

Beyond V1, there are a number of extrastriate areas in both mice\cite{Andermann2011, Marshel2011, Juavinett2017} and rats\cite{Espinoza1983RetinotopicRat, Coogan1993}. Anatomical evidence suggests a hierarchical organization\cite{Wang2007, Wang2011GatewaysCortex}, as the connectivity patterns of extrastriate area are suggestive of homology with the dorsal and ventral pathways in primate cortex. Over the last decade, a growing number of functional studies have ventured further into extrastriate cortex, and have begun to systematically characterize basic response properties of these visual areas in the mouse\cite{Andermann2011, Marshel2011, Glickfeld2013, Glickfeld2017, DeVries2019, Siegle2021}. Visual areas corresponding to a putative dorsal pathway in mice contain neurons that are particularly tuned for motion\cite{Andermann2011, Marshel2011, Glickfeld2013} and exhibit sensitivities to motion processing in specific portions of the visual field\cite{Sit2020}. In the putative ventral stream, several studies have found less motion sensitivity and greater tuning for higher spatial frequencies\cite{Glickfeld2013, Tohmi2014}, properties that would facilitate visual shape processing. 
% 
Despite the many similarities between rodent and primate visual systems, there are also striking differences. For example, primates have a fovea, a specialized region that takes up a tiny portion of the retina but contains the highest density of photoreceptors\cite{Perry1985}, while rodents do not. Primates use foveal vision for high contrast, high acuity tasks, such as a reading or detecting small objects. Within V1, both species exhibit the hallmark property orientation tuning, but differ in spatial organization: in primates, cells preferring a particular orientation are organized in a ``columnar'' structure\cite{Blasdel1986}, while in rodents, orientation selectivity forms a salt-and-pepper organization\cite{Ohki2005}. In addition, in both rats and mice, projections from V1 extend to areas that process motor and other non-visual information\cite{Wagor1980RetinotopicMouse, Chen1994, Brecht2004}, while in the primate visual system, extrastriate or higher-order areas project to regions that process non-sensory information. 

% ----------------------------------------------------
% Rodent Models for Vision
% ----------------------------------------------------
\section{High-Level Vision in Rodents}
In traditional visual neuroscience, mice were ignored for a long time under the assumption that they do not exhibit interesting visual behaviors --- however, the immense power of their experimental and genetic tractability has renewed interest in exploring the extent and range of their visual capacities, such as visually-guided hunting behavior\cite{Hoy2016, Meyer2020, Michaiel2020}. In contrast, rats stand out among rodents because there is a long and rich history of studies of their visual behavior. In the 1930s, Lashley and others published a series of studies demonstrating the ability of rats to perceive and discriminate visual shapes\cite{Lashley1912, Lashlsey1930, Lashley1938}. In the 1970s, a landmark study by O'Keefe and colleagues demonstrated the visually-guided spatial navigation\cite{OKeefe1971}. Compared to mice, rats are also not strictly nocturnal, and can be found hunting during the day (rats are actually predators of mice), have higher visual acuity\cite{Prusky2000}, and perform better and learn quicker on many visual and cognitive tasks\cite{Whishaw1995, Wishaw1996}. In the context of visual object recognition behavior, several groups have since built upon these studies, and have shown that rats rely on a range of visual strategies to perform object recognition tasks\cite{Zoccolan2009, Tafazoli2012Transformation-TolerantPriming, Vermaercke2012, Alemi-Neissi2013MultifeaturalRecognition, Vinken2014}.

% Specific to object recognition 
Importantly, recent studies have also identified a collection of visual areas along the lateral, posterior edge of rat cortex that may exhibit several features that have been considered hallmark characteristics of the primate ventral pathway\ref{rat_visual_areas}. Specifically, starting from primary visual cortex (V1), these areas area the lateromedial area (LM), the laterointermediate area (LI), laterolateral area (LL), and the occipitotemporal
cortex (TO). In support of anatomical evidence suggesting a hierarchical progression of low to high across these areas\cite{Coogan1993, Wang2012NetworkCortex, DSouza2020CanonicalHierarchy}, a growing body of functional evidence has identified several hallmark features of neurons in these visual areas of the rat that are reminiscent of important properties of the primate ventral pathway. For example, studies have found an increase in stimulus-evoked response latency across these areas\cite{Vermaercke2014, DSouza2020}, suggesting a sequential processing of stimulus information.

In the context of visual object recognition, there are several hallmark features of the primate ventral pathway that have been found, to varying degrees, in rat lateral cortex. The most consistent of these is an increase in receptive field size, from V1 to LM, and on\cite{Vermaercke2014, Tafazoli2017}. Less clear, is the extent to which shape selectivity and transformation tolerance across rat lateral cortex clearly maps onto areas of the primate ventral path. For example, several groups have found that more lateral areas are better at decoding object identity than V1, even across identity-preserving transformations\cite{Tafazoli2017, Vermaercke2014, Froudarakis2020}. However, in contrast to the primate ventral path, Vermaercke and colleagues found that this increasing discriminability was only found when moving objects were used\cite{Vermaercke2014}. Moreover, in contrast to the primate visual system, the same group also found an increasing selectivity for drifting gratings. In another study, Vinken and colleagues found that although areas LI and LL did a better job at discriminating natural images than scrambled images, they did not find category-selective representations, as might be expected in a  higher-level area of visual cortex\cite{Vinken2016}. 


% Not enough tools in rats, even tho their behavior is awesome
\section{Goals of the present study}
The many similarities between monkey and rodent visual systems, from low-level areas like V1 to primate IT and putative higher-level rodent counterparts, such as LI or LL, suggest that rodent extrastriate areas can serve as a valuable complement to monkey visual models --- however, many aspects of the rodent visual pathway remain to be characterized in detail. Although a large body of scientific research, including visual object recognition, relies on the rich behavioral repertoire of rats, it has proved to be more challenging to fully leverage recent developments in optical imaging and genetic tools in the rat, as it has been done with great success in smaller animals like mice\cite{Luo2018}. Notably, all prior studies of neural processes underlying object recognition in rats have relied on single-unit responses and acute electrophysiology in passively viewing or anesthetized animals. Few or no paradigms exists for large-scale cellular resolution imaging of extrastriate cortex in awake, head-fixed rats (though see V1 recordings in \cite{Greenberg2008}). 

One of the biggest challenges in applying cellular resolution optical imaging approaches to rodents has been head-restrained preparations, which have become relatively standard in studies of mouse visual cortex. While head-fixed preparations are routinely used in mice, rats are much harder to restrain, given their larger size and prohibitive movement artifacts for cellular-resolution imaging. As such, most studies using head-fixed rats rely on electrophysiology, or freely moving approaches with limited optical access\cite{Scott2013}. Furthermore, the lateral position of candidate object recognition areas makes optical access all the more challenging, and rat imaging studies have been restricted to medial V1\cite{Greenberg2008, Ohki2005}. Nonetheless, head-fixed preparations afford certain experimental advantages, such as restrained movement, precise stimulus control, and longitudinal studies that can track the same neurons over long timescales. 

% Given likely differences in response characterizations between optical imaging and and electrophysiology approaches\cite{SiegleReconcilingElectrophysiology}, it is challenging to interpret results and attribute possible differences to methods as opposed to species-specific adaptations. 

% Even among rodents, there may be species-specific differences in how analogous brain areas are to parts of the primate ventral stream. For example, in mice, areas LM and AL seem to contain the most object-specific, view-tolerant information\cite{Froudarakis2020}, while studies in rats point to areas LI, LL, or even TO\cite{Vermaerke2014, Tafazoli2017}. Such differences may be due to behavioral differences in visual capacities between rats and mice, \textit{e.g.}, visual acuity\cite{Prusky2000}, how diurnal or nocturnal they are, and the extent to which they rely on visually-guided behavior\cite{Whishaw1995, Whishaw1996}. Moreover, even among rat studies, paradigms differ in stimulus set, recording techniques, and behavior states. 

% PRESENT STUDY -- what am i doing and why.
The recent emergence of mouse as a powerful system for studying visual circuits has largely been drive by the ability to apply a wide array of powerful genetic tools for circuit dissection, along with technological advancements that allow simultaneous acquisition and manipulation of large neuronal populations in awake animals. On the other hand, despite the long history of rich behavior studies in the rat, and its continued popularity and value as a model system in many other fields of biomedical research, many of those existing tools have been leveraged in rats. An overarching goal of the present study is to combine these two threads. Here, we aim to bridge the gap between neural circuit access and rich behavioral capacities in the rat. We demonstrate reliable, high-throughput cellular resolution optical imaging in awake, head-fixed rats for the first time, and show the feasibility of applying the powerful tools available in systems neuroscience to a rodent model of high-level vision. 

In Chapter 1, we develop a system for automated, high-throughput training of visual behaviors in rats. This works build off of previous work showing visual object recognition capacities of rats, and is inspired by the many high-throughput behavior systems developed for testing non-vision-specific tasks in rodents. We describe an open-source, modular, and high-throughput system in which large cohorts of animals can be trained on complex visual behaviors. In this chapter, we also describe rats' perceptual choices in response to visual stimuli for which we will characterize neural responses in later chapters.

In Chapter 2, we describe our efforts to engineer systems for optical imaging in head-fixed rats. As the first demosntration of 

the functional organization of a subset of visual areas for the first time with cellular resolution imaging in awake, head-fixed animals. 

We describe the macro- and micro-scale organization of V1, LM, and LI in the context of a range of stimulus classes. Specifically, we characterize feature tuning and object tuning in rat visual cortex to examine how neural circuits encode feature variations that either preserve object identity or alter object identity. 

The aim of the present study was to expand on existing characterizations of response properties in rat extrastriate visual cortex using calcium imaging.


In Chapter 1, 



The present work aims to fill gaps in our current understanding of the rodent visual pathway that will further enhance the rodent as an animal model that leverages greater experimental accessibility to more deeply probe fundamental questions about how visual information is organized and processed in the mammalian brain. Taken together, this work offers the potential for a new, powerful, complementary animal model for the study high-level vision.



%%%% 

% Vision is critical for guiding behavior in species throughout the animal kingdom. Though nonhuman primates have been the classic model for vision, alternative systems offer a way to reveal general principles of visual processing as well as species-specific adaptations. Advances in imaging techniques and genetic tools have made genetically accessible animals, like mice, valuable systems for understanding neural circuits of vision. In contrast, although rats have long been studied for visually-guided behaviors, large-scale cellular resolution access to their visual cortex has proved to be far more challenging. Here, we aim to bridge the gap between neural circuit access and rich behavioral capacities in the rat. We characterize the functional organization of a subset of visual areas for the first time with cellular resolution imaging in awake, head-fixed animals. 