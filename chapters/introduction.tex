\chapter{Introduction}
\label{introduction}

% hierarchy, vision, Hubel & Wiesel, 1962, J Physiology
% We perform extraordinary behaviors every day. From making a simple decision about what to eat for breakfast or finding keys under a clutter of mail, to playing a boardgame or planning a family BBQ --- each mindless or mindful task is a complex behavior produced by the activity of many individual neurons in the brain. A primary goal of systems neuroscience is to explain how behaviors arise from the combined activity of these neural populations.

% Interesting examples
The physical world poses a diversity of problems that all animals must solve --- where to find food, where to seek shelter, and how to avoid predators. Solving these problems requires cognitive capacities. Broadly, cognition involves processing one's the environment and using that information to make decisions and take actions based on the available information. Under this framing, cognitive capacities include spatial navigation, remembering past events and places, or evaluating quantity and quality to make a decision about what to do.  

Examples of cognitive behavior abound throughout the animal kingdom. In the Saharan Desert, foraging desert ants (\textit{Cataglyphis fortis}) leave their nest and travel distances thousands of times their body length to find food. In this vast and featureless environment, they rely on dead reckoning to track how far and in what direction they have traveled in order to find a return path home\cite{Muller1988, Wehner2003}. When honey bee foragers (\textit{Apis mellifera}) discover a rich food source, they perform a waggle dance back at the hive that communicates to the other foragers the location and quality of the food\cite{vonFrisch1974, Dyer2002}. In caching birds, there are many examples of evidence of long-term memory and episodic memory. Black-capped chickadees (\textit{Parus atricapillus}) can recover caches up to 28 days after caching\cite{Hitchcock1990}. Scrub jays (\textit{Aphelecoma californica}) appear to remember not just where they cached, but also the quality of their cache: after caching both nuts (a stable food) and waxworms (a decaying food source), they will recover more waxworms after a 4 hour delay (as waxworms are their preferred food), but after a 5 day delay, they will recover more nuts\cite{Clayton1998} --- this distinction demonstrates memory for not just what and where they cached, but also when, hallmark signatures of episodic memory.

Our knowledge of the world is based on how we perceive the world --- perception is a constructive and interactive process that relies not only on a physical stimulus, but also on its processing by a perceiver. Understanding a behavior, then, is not just knowing about the physical inputs and the observable outputs, but also the process by which sensory information is transformed into perception and action. The effortlessness of many of our daily behaviors, grabbing a favorite mug out of the cupboard, often belies the complexity of the problem. The brain transforms a flurry of inputs hitting the retina (say, the particular arrangement of photons when I look at the mug) into a coherent, recognizable object (I recognize the same mug in the cupboard or on the breakfast table). The latter is a high-level representation that is stable across different views and moments in time. Implicit in this framing, the idea of transforming photoreceptor activity into the perceiver's recognition, is that each percept or action has a representation of information in the form of patterns of neural activity. 

The present study approaches questions of cognitive behavior from the lens of visual perception --- not necessarily due to an inherent primacy of vision (though in the case of primates, vision is indeed the principle sensory modality), but due to the wealth of research, spanning philosophy, psychology, and modern neuroscience, that has gradually built ever-evolving frameworks of how we see as a guiding framework for understanding how we, as perceivers and actors, interact with and navigate our environments. Within perception, visual recognition of objects represents just one of many computational puzzles that the brain has solved over evolutionary timescales, and one that modern day neuroscience still seeks to understand. Understanding core principles of visual perception promises to reveal fundamental insights of vision, as well as of cognitive processing, in general. 

\section{Visual cognition and object recognition}
% Intro visual cognition
In 1949, in one of the most influential texts for neuroscience, \textit{The Organization of Behavior}\cite{Hebb1949}, Donald Hebb laid out the perplexing puzzle of ``the generalization of the perception of patterns,'' citing the work of \citet{Lashley1938}:  ``Man sees a square as a square, whatever its size, and in almost any setting. A rat trained to look for food behind a horizontal rectangle with after choose almost \textit{any} horizontal figure... Trained to choose a solid upright triangle and to avoid an inverted triangle, he will discriminate consistently between outlines of triangles, triangles with confusing figures added (such as circumscribing circles), and triangles of different size, which cannot thus excite the same retinal cells simultaneously'' (p.14). This capacity, the ability to recognize a thing across the many ways it can appear, is called invariant object recognition, or more precisely, transformation-tolerance object recognition. More broadly, visual object recognition represents a quintessential formulation of the question of cognitive behavior:  how is the activity from an effectively infinite number of possible impingements by physical stimuli on one's sensory receptors (for example, photoreceptors in the retina) transformed into the activity of (which and how many) neural populations that are the cause of the perceiver's experience recognizing a given object?

Progress in biological research relies on linking phenomena at different levels of abstraction, from macroscales of behavior (behavioral success on a benchmark test, perception of a visual illusion), to correlated neural activity (neural response properties during a given stimulus or behavior), and still further to mechanistic understanding (the activity of particular cell types and their intracellular processes). A primary goal of systems neuroscience is to explain how behaviors arise from the combined activity of these neural populations, thus to span across these broad levels of explanatory power. While many behavioral phenomena can be quantified and parameterized in almost any observable animal, access to the neural processes that process and transform incoming stimuli to give rise to the observed behavior is much more limited in scope. 

% Ok, so what do we know?
Accessing neural activity during behavior is always challenging. Neurological and neuropsychological data were critical for identifying and specifying component processes of various cognitive systems. The first quantitative study of visual object recognition deficits in patients with brain lesions was by Brenda Milner in 1958, where she reported that patients with right temporal lobe lesions had difficulty identifying anomalies in tests of visual scenes\cite{Milner1958}. This discovery was followed by many other reports of specific deficits in visual tasks for patients with cerebral lesions, including facial recognition or identifying objects from line drawings\cite{Warrington1967,Newcombe1969}. 

In addition to inferring function from lesion studies in humans, early efforts to record neural activity in the context of perceptual and cognitive behavior focused on monkeys. For decades, nonhuman primates, like the macaque (\textit{Macaca}), were the dominant animal model for studying cognitive behavior. This makes sense, given their similarity to humans and the traditional view that studying brains similar to our own will lead to fundamental insights about the human brain. In the 1970s, Mountcastle and Evarts developed a method to record from single-units in awake, behaving monkeys with electrodes --- their work, and the work of many others around that time, were the first demonstrations of correlations between cognitive processes, such as perception or voluntary action, and patterns of neural activity\cite{Mountcastle1975,Evarts1968,Hubel1968}. 

% cortex
These early studies in neuropsychology, neurology, and cognitive neuroscience launched decades of subsequent research that has since identified the cortex, the brain's layered outer structure of neural tissue, as critical for cognitive behaviors. Although research in other animal models, such as birds and bees, points to cognitive capacities in the absence of a strictly defined 6-layer mammalian neocortex, in mammals, it is indispensable for many important cognitive behaviors, such as learning a skill\cite{Kawai2015}, remembering particular places or things\cite{Squire1991}, and basic perception\cite{Tanaka1991,DiCarlo2012}. 

% ----------------------------------------------------
% Primate visual system
% ----------------------------------------------------
% What do we know about visual areas (properties)
\section{Functional organization of the visual system}
% visual system/visual cortex
Vision is arguably the best understood modality in the mammalian brain today. The visual cortex, where most of the processing for visual perception is thought to occur in mammals, takes up about half of the macaque brain\cite{Felleman1991}. Research in primates and carnivores using single-unit recordings has led to the discovery of many fundamental properties of visual cortex, how neurons responds to particular types of visual stimuli\cite{Hubel1962,Hubel1968,Reid1991}. 

In primates, visual cortex is traditionally divided into two parallel pathways: the ``ventral'' pathway runs along the ventral side of the occipital and temporal lobes, and is associated with visual shape processing, while the  ``dorsal'' pathway runs along the dorsal side of the occipital and parietal lobes, and is associated with motion processing and spatial relationships~\cite{Mishkin1982, Ungerleider1994WhatBrain, Felleman1991}. Early evidence for separate processing streams comes from studies in which lesions in different parts of these cortical paths produced contrasting effect in monkeys. For example, lesions of the inferior temporal (IT) cortex cause severe deficits in visual discrimination tasks, but have no effect on visuospatial tasks, such as visually-guided reaching or relative judgements of visual distances. In contrast, parietal cortex lesions, along the dorsal pathway, do not affect performance on visual discrimination, but severely disrupt visuospatial behaviors. As such, the dorsal path is referred to as the ``where'' pathway (concerned with \textit{where} an object is), and the ventral path is often called the ``what'' pathway (concerned with \textit{what} an object is)\cite{Ungerleider1994WhatBrain}. These pathways have also been framed in terms of a perception/action dichotomy --- that is, recognizing an object versus reaching toward it\cite{Goodale1992}.

While the two-stream framework is certainly an oversimplification, what we know is that the primate visual cortex extracts and transforms information from one area to the next. The process begins at the retina, where photoreceptors transduce photons into neural activity, which then travels through a series of brain regions, from the dorsal lateral geniculate nucleus (dLGN) and on to striate or primary visual cortex (V1), and beyond. Each step in this process extracts increasingly refined features, from neurons in V1 that selectively responds to edges of particular orientations, through successive visual areas beyond V1, collectively called ``extrastriate'' cortex --- here, neurons respond selectively to higher-order visual features, such as objects, faces, or motion\cite{Orban2008, DiCarlo2012}. 

Visual object recognition in primates is thought to emerge through these sequential processing stages across the ventral visual pathway~\cite{Rust2010SelectivityIT, DiCarlo2007, DiCarlo2012}. From early to late stages of the hierarchy, single neuron properties vary systematically. In V1, receptive fields are smaller and cells are optimally stimulated by oriented edges or gratings\cite{Hubel1968}. In subsequent stages, receptive field sizes get larger, while selectivity for complex shapes increases~\cite{Desimone1984, Logothetis1996}. At the same time, neurons in later, higher-level areas exhibit tolerance to identity-preserving transformations: in these highest stages of the ventral pathway, inferotemporal or IT cortex, cells selectively respond to a given object regardless of changes in size, location, or particular appearance on the retina~\cite{Tanaka1996, DiCarlo2012}. Some studies even report neural activity highly specific for behaviorally-relevant categories, such as faces\cite{Kanwisher1997, Tsao2006} and scenes\cite{Epstein1998}.

% We know that visual cortex successively processes input in a hierarchical cascade: lower-level areas of visual cortex tend to represent local features of the retinal image, \textit{e.g.}, oriented edges in local image patches, while subsequent higher-level extrastriate areas represent increasingly more abstract properties of the external world, \textit{e.g.}, object shape and identity. The hierarchical organization of the visual system is thought to play a key role in its ability to extract and learn about latent structure from sensory inputs. Over the past 50 years\cite{Hubel1962}, the idea of a hierarchically organized system has framed our understanding of how complex behavior arises. In practice, it has also inspired powerful multi-layered computational networks that are capable of impressive feats\cite{Riesenhuber1999, Krizhevsky2012}.

From careful quantification and parameterization of visual phenomena and single-unit responses, influential models of visual processing and cortical function have laid out possible mechanisms for neural computations\cite{Riesenhuber1999,Ferster2000,Carandini1994,Carandini2012,DiCarlo2012}. However, due to the complexity and scale of the primate brain and practical difficulties of nonhuman primate research, experimental progress toward a more mechanistic understanding of cortical computations in the primate brain has been slow.  

% What's holding us back!
% While much work on the visual system has been done studying the responses of single neurons, we lack a complete understanding of how populations of neurons work together to represent the external world. 
Mechanistic understanding of visual cortical computations has been partially limited by the tools at our disposal to probe neuronal populations. Decades of research in the primate visual system have relied on investigators sampling from neuronal populations in the ventral visual pathway of monkeys using single microelectrodes or electrode arrays, but these techniques have several fundamental limitations. First, they are only able to sample from a comparatively small number of neurons at time, generally requiring the assembly of serially-sampled ``pseudo-populations'' in order to explore population coding. Second, electrophysiological recording techniques typically do not allow the same populations of neurons to be studied over long periods of time. In some cases, a small number of neurons can apparently be isolated for days in a chronic preparation; however, it is usually impossible to be certain that these cells are the same across days. Third, electrophysiological recordings are blind to particular cell-types, thus precluding the use of many powerful methods for targeting neural populations, and identifying which classes of cells partake in which types of computations.

% In contrast, rapid advancements in powerful molecular and genetic tools have made it possible to study cortical function at a more mechanistic level than ever before\cite{Luo2008, Luo2018} --- at least in a subset of animal models. Among mammalian models, this has borne out dramatically in the study of mice for sensory processing. Genetic strategies in mice allow experimenters to target and manipulate the activity of specific, genetically-defined classes of cells\cite{Kerlin2010, Atallah2012, Adesnik2012}, identify its connections to other neural populations, and visualize hundreds of neurons in an an awake, behaving animal\cite{Andermann2010,Chen2012,Poort2015,Burgess2017,Dana2019}. 

% Furthermore, recent advances in population recordings, which allow the activity of many neurons to be captured at the same time, have dramatically expanded the scope of questions that researchers are able to ask. Extracellular electrophysiology and two-photon optical imaging are the most popular methods for recording neural populations, due to their single-cell resolution, relative ease of use, and scalability. Within the last five years, large-scale recordings of hundreds and even thousands of neurons are becoming increasingly standard using electrophysiology \cite{Steinmetz2019, Siegle2021} and optical imaging \cite{Stringer2019a, Weisenburger2019, Sofroniew2016}. 

In non-primate models, developments in both molecular and genetic tools and in recording techniques have fundamentally changed the nature of questions that systems neuroscience can ask. Extracellular electrophysiology and two-photon optical imaging are the most popular methods for recording neural populations, due to their single-cell resolution, relative ease of use, and scalability. Within the last five years, large-scale recordings of hundreds and even thousands of neurons are becoming increasingly standard using electrophysiology \cite{Stringer2019,Steinmetz2019,Siegle2021} and optical imaging \cite{Sofroniew2016,Stringer2019geometry, Weisenburger2019}. For molecular access to neural populations, genetic strategies allow experimenters to target and manipulate the activity of specific, genetically-defined classes of cells\cite{Kerlin2010, Atallah2012, Adesnik2012}, identify its connections to other neural populations, and visualize hundreds of neurons in an an awake, behaving animal\cite{Andermann2010,Chen2012,Poort2015,Burgess2017,Dana2019}. 

However, primates are a difficult animal system in which to use existing tools for probing neuronal populations, such as \textit{in vivo} 2-photon calcium imaging \cite{Ohki2005}: monkey cortex is 2.5mm thick on average\cite{Koo2012}, making it difficult to image much beyond layer II, due to optical scattering limits. In addition, monkeys are capable of significant head movement even when head-fixed, and even at rest their brains move significantly due to pulsatile motion, which complicates imaging. At the same time, the complexity of the monkey ventral visual pathway, albeit more similar to the human visual system, is more challenging to understand relative to simpler systems. As a result of all of these factors, a growing number of investigators have turned to simpler alternative models to explore a variety of ``high-level'' sensory and cognitive functions in less complex systems\cite{Brunton2013, Miller2017TwoStep, Aronov2014, Glickfeld2017}).


% ----------------------------------------------------
% Non-primate models for cognition
\section{Rodent models of cognition}
% Rodent models.
% ----------------------------------
% Rats better than primates for behavior stuff -- the new model of cog behaviors
The laboratory rat \textit{Rattus norvegicus} was the first mammalian species domesticated for scientific research\cite{Jacob1999}, and has since been the most widely studied species in biomedical research. Rats are not only experimentally tractable and straightforward to raise in lab environments, but moreover, they are extremely adept at learning laboratory tasks. Since the first half of the 20th century, rat behavior has been carefully quantified and parameterized for a huge range and diversity of behaviors. Starting from 1912, Karl Lashley published a series of influential papers on visual shape discrimination in rats, eventually showing evidence that rats are able to generalize recognition of shapes across various transformations in size or surrounding contetxt\cite{Lashley1912, Lashley1930, Lashley1938}. Donald Hebb developed classic assays for testing spatial navigation and other cognitive capacities in rats\cite{Hebb1946}. In the 1970s, a landmark study by O'Keefe and colleagues demonstrated visually-guided spatial navigation\cite{OKeefe1971}. Since then, training paradigms and tests of cognition have continued to be developed for rats, and today, rats are standard models for studying a range of cognitive behaviors, such as decision-making\cite{Raposo2012, Miller2017TwoStep, Piet2018, Brunton2013}, working memory\cite{Bratch2016, Fassihi2014, Akrami2018}, and spatial navigation\cite{OKeefe1971, Whishaw1995, Aronov2014, Poo2020}. 

Several technological advancements have maintained rats as the preferred rodent model of choice for many behaviors, cognitive or otherwise. First, with the advent of computer-controlled experiments and the increasing capacity for low-cost and high-throughput experimental techniques, it is possible to train many rats at once. Compared to the 2-3 monkeys used in traditional cognitive neuroscience, one can train >50 rats on a given behavior task\cite{Brunton2013, Constantinople2019}. This has largely been pioneered by groups studying decision-making, action-planning, and working memory\cite{Raposo2012,Brunton2013,Miller2017TwoStep,Piet2018,Akrami2018, Constantinople2019}. Second, with the development of smaller, more efficient electrode arrays, combined with computational power for acquiring and analyzing massive data sets of hundreds of neurons\cite{Pachitariu2016, Steinmetz2018,Stringer2019}, it is now possible to record from large neural populations within and across brain areas while the rat performs a behavior task.

Third, a growing array of genetic tools makes it possible to target genetically-identified populations of cells\cite{Tomita2009,Witten2011,Igarashi2018}. These developments are critical for dissecting neural circuits of behavior at a mechanistic level. Genetic tools allow targeted expression of genes for controlling cell-type specific populations, or even sub-cellular localization of expression, while animals are engaged in a task, and are thus immensely powerful for identifying cellular mechanisms of neural computations, and ultimately, the animal's behavior. Furthermore, it is also possible to combine targeted neural manipulation with neural recordings. For example, ``optrode'' arrays combine electrophysiology and optogenetics in rats expressing a particular gene in a subset of cells:  an electrode array records from neurons and an optical probe can control genetically-identified neurons that express light-sensitive proteins that silence or activate their activity when stimulated\cite{Gradinaru2007,Yang2018}.
% but same issues apply in rats as monkeys (ephys, no imaging).

% Genetic stategies
% In contrast, rapid advancements in powerful molecular and genetic tools have made it possible to study cortical function at a more mechanistic level than ever before\cite{Luo2008, Luo2018} --- at least in a subset of animal models. 

Among mammalian models, nowhere has the combined power of genetic tools and techniques for population access borne out as dramatically as in mice. Of note, the combination of genetic tools with optical techniques for cellular resolution imaging and optical manipulation has proved immensely powerful. It is possible to record from many hundreds of neurons while animals are engaged in a task\cite{Andermann2010,Poort2015,Burgess2017}. Even more recently, it is now possible to record from tens to hundreds of thousands of neurons at once\cite{Stringer2019geometry,Weisenburger2019,Stringer2021precision} or use holographic light stimulation to manipulate specific subsets of active cells in the context of a learned behavior paradigm\cite{Chong2019, Gill2020}. With the mouse, and unlike with any other mammalian model, it is possible to leverage the immense power of molecular biology to selectively record and manipulate the activity of large populations of cells with incredible precision. Importantly, mechanistic studies of cortical processing are now possible in a mammalian model. While most of the focus has been on visual cortex (reviewed below, and also see\cite{Huberman2011,Glickfeld2017,Niell2021}), the striking similarity in microcircuit organization between cortical areas has long been suggestive of canonical cortical computations across modalities in cortex and even across species\cite{Miller2016}. 

In rats, cellular resolution imaging techniques have been notoriously difficult to apply with regularity. One of the biggest challenges has been head-restrained preparations. Rats are much bigger than mice, and their larger size makes them harder to restrain. On a practical level, rats can easily rip themselves out of the head-restraining implants. Laboratory tasks used to study cognitive behaviors take many weeks and month to train --- to make this worthwhile, it is highly desirable to record from many hundreds of neurons per animal, and without the risk of losing the animal due to unreliable head-restraint. Even when rats do not rip themselves out of the implants, they can produce prohibitively large artifacts for cellular resolution imaging. As such, most studies using head-fixed rats rely on electrophysiology, or freely moving approaches with limited optical access\cite{Scott2013,Scott2018}. Although more recent developments in electrophysiology techniques allow for much larger population sizes\cite{Steinmetz2018}, electrophysiology approaches are inherently limited in both the sheer number and cell-type specificity of neural populations that can be characterized or manipulated over time. 

Given the importance of cortex or cortex-like structures in cognition, combined with the experimental and genetic tractability of mice, it is alluring to move away from existing models of cognitive behavior, namely, nonhuman primates and rats, and turn to mice. Unlike monkeys or rats, mice were long ignored by traditional neuropsychologists and cognitive neuroscientists due to the assumption that mice do not perform interesting cognitive behaviors. Around the same time that Lashley showed evidence of pattern discrimination in rats, others had attempted but failed to show similar capacities in mice\cite{Yerkes1907,Waugh1910}. However, in more recent years, a growing number of studies have begun to explore cognitive behavior in mice. Within the past 10 years, these studies have demonstrated that contrary to prior beliefs, mice can be trained to perform a variety of sensory and cognitive tasks, such as decision-making and evidence accumulation\cite{Morcos2016,Pinto2018, Krumin2018,Lee2020}, working memory\cite{Light2010}, and spatial navigation\cite{Dombeck2007,Harvey2009}. 

% But rats are great, and rich history...
However, rats have a rich history of cognitive and perceptual behavior studies that span over 100 years. Moreover, with recent developments in rat genetics, it is possible to engineer the rat genome with unprecedented efficiency\cite{Aitman2008,Geurts2009,Huang2011rats,Tesson2011,Dayton2018}, and to produce genetically modified rat strains for basic neuroscience research\cite{Witten2011,Scott2013,Igarashi2018}. More broadly, rats have been the most widely used and preferred model in biomedical research, including substance abuse disorders, developmental disorders, such as autism spectrum disorders, and cardiovascular diseases\cite{Aitman2008,Ellenbroek2016,Homberg2017}. For example, relative to mice, rats are larger, which makes certain procedures easier to perform, exhibit social behaviors\cite{Viana2010,Bartal2014,Netser2020}, and are generally faster and easier to train\cite{Colacicco2002,Jaramillo2014}.

% Ephys:
% (Jun et al., 2017 (neuropixels); 
% Siegle et al., 2019 --Nature, 2021,. (hierarhcy, Nature)
% Stringer et al., 2019a -- spontaneous bebaviors, Science

% ca imaging: 
% (Sofroniew et al., 2016; -- meoscope, elife
% Stringer et al., 2019b --high-dim geometry, Nature
% Weisenburger et al., 2019). --- alipasha, Cell

% ----------------------------------------------------
% Rodent Models for Vision
% ----------------------------------------------------
\section{The rodent visual system}
Early work in the first half of the 20th century by Lashley and others showed that rats can perceive and discriminate visual shapes\cite{Lashley1912, Lashley1930a, Lashley1938}. More recently, several groups have built upon this work, and have shown that rats, and even mice, can perform invariant object recognition tasks with complex object shapes\cite{Zoccolan2009,Tafazoli2012,Vermaercke2012,Alemi-Neissi2013,Vinken2014,Djurdjevic2018, Froudarakis2020}.

As described above, an important driving force behind the emergence of rodents as models for cortical circuits has been the widely available and rapidly growing array of genetic tools for analyzing connectivity and probing and controlling activity in neural circuits\cite{Luo2008, Luo2018}, and access to large populations of neurons\cite{Sofroniew2016, Stringer2019geometry, Weisenburger2019,Stringer2021precision}. Beyond these powerful tools, rodents also offer excellent experimental accessibility. They have a lissencephalic cortex, which means that their cortex is smooth and does not have the characteristic folds of gyri seen in primate cortex. This is particularly advantageous for optical imaging techniques, as more of the brain is accessible right at the surface. Second, rodents cost much less than primates to keep in the laboratory and are easier to house in large numbers, which means access to larger sample sizes in a given experiment. Third, due to their smaller size, rodents can be trained on even time-consuming and laborious behavior tasks, as many animals can be trained in parallel.

%Evidence that rats are good at stuff
On the other hand, mice and rats are have much lower visual acuity, which raises the possibility that their visual system is fundamentally different from the visual systems of primates. However, in addition to behavioral evidence of complex visual cognition in rodents, increasing evidence suggests that rodents possess rather sophisticated visual machinery that would make them a tractable model for studying multi-level visual processing. Many basic properties of visual function, at least from the retina and up to V1, are present in rodent visual cortex\cite{Huberman2011}. Within V1, several circuits underlying a range of cortical computations have been found, including orientation selectivity \cite{Ko2011, Lien2013}, surround suppression\cite{Adesnik2012}, and gain control\cite{Atallah2012}. 

Beyond V1, rodents have a number of extrastriate areas\cite{Andermann2011, Marshel2011, Juavinett2017,Espinoza1983, Coogan1993}. Anatomical evidence suggests a hierarchical organization based on connectivity patterns of extrastriate cortex\cite{Wang2007, Wang2011}, and more recent studies provide growing evidence for functional specialization of both discrete visual areas and processing pathways\cite{Andermann2011, Marshel2011, Glickfeld2013, Glickfeld2017, Beltramo2019, DeVries2020, Siegle2021, Blot2021}. Visual areas corresponding to a putative dorsal pathway in mice contain neurons that are particularly tuned for motion\cite{Andermann2011, Marshel2011, Glickfeld2013} and exhibit sensitivities to motion processing in specific portions of the visual field\cite{Sit2020}. In the putative ventral stream, several studies have found less motion sensitivity and greater tuning for higher spatial frequencies\cite{Glickfeld2013, Tohmi2014}, properties that would facilitate visual shape processing. 

% Specific to object recognition 
In the context of neural circuits underlying visual object recognition, recent studies have also identified a collection of visual areas along the lateral, posterior edge of rat cortex that may exhibit several features that have been considered hallmark characteristics of the primate ventral pathway. Specifically, starting from primary visual cortex (V1), these areas area the lateromedial area (LM), the laterointermediate area (LI), laterolateral area (LL), and the occipitotemporal cortex (TO). In support of anatomical evidence suggesting a hierarchical progression across these areas\cite{Coogan1993, Wang2012NetworkCortex, DSouza2020}, functional evidence has identified several hallmark features of neurons in these visual areas of the rat that are reminiscent of important properties of the primate ventral pathway. For example, studies have found an increase in stimulus-evoked response latency across these areas\cite{Vermaercke2014, DSouza2020}, suggesting a sequential processing of stimulus information, increasing receptive field sizes\cite{Vermaercke2014, Tafazoli2017}, and increasing shape selectivity and tolerance of neural selectivity to identity-preserving transformations\cite{Vermaercke2014, Vermaercke2015, Tafazoli2017, Matteucci2019b, Froudarakis2020}.

% BUT, rodents diff from primates.
Despite the many similarities between rodent and primate visual systems, there are also striking differences. For example, primates have a fovea, a specialized region that takes up a tiny portion of the retina but contains the highest density of photoreceptors\cite{Perry1985}, while rodents do not. Primates use foveal vision for high contrast, high acuity tasks, such as a reading or detecting small objects. Within V1, both species exhibit the hallmark property orientation tuning, but differ in spatial organization: in primates, cells preferring a particular orientation are organized in a ``columnar'' structure, while in rodents, orientation selectivity forms a salt-and-pepper organization\cite{Ohki2005}. The organization of rodent visual cortex also appears to be a shallower network, with less distinct functional separation from area to area\cite{Oh2014,Gamanut2018,DSouza2020}. Moreover, the rodent visual system has specialized visual pathways, evidence for which has yet to be described in detail for the primate visual system. According to the standard, primate-based model of visual processing, all visual information flows through the retino-geniculate pathway to V1, the gateway to higher-order visual areas. In contrast, recent studies in mice have discovered a specialized pathway through a sub-cortical nucleus, the superior colliculus (SC), that bypasses V1 altogether to target a higher-order visual area, the postrhinal cortex (POR)\cite{Beltramo2019}. Additional studies have discovered specialized pathways through other sub-cortical nuclei, including the lateral posterior nucleus (LP) in the rodent visual system (called the pulvinar in primates), which appear to carry distinct visual information than what is relayed by cortico-cortical pathways\cite{Bennett2019,Blot2021}.


In short, comparative approaches between rodents and primates promise to provide valuable insight into general principles of cortical function and organization, as well as species-specific adaptations specialized for the ethological demands of each animal. Among rodents, mice and rats are the most widely used models, and there are widely available tools that can be applied in both species. Although the history of cognitive behaviors is more recent in mice, we are quickly learning more about the extent and range of their visual behaviors, such as visually-guided hunting and prey-capture\cite{Hoy2016, Meyer2020, Michaiel2020}, and other cognitive behaviors, such as task switching\cite{Lee2020} and evidence accumulation during decision-making\cite{Pinto2018}. Relative to rats, however, mice often take longer to train and perform worse on certain tasks, such as water mazes, a popular method for studying spatial navigation and visually-guided behavior\cite{Whishaw1995, Whishaw1996,Colacicco2002,Jaramillo2014}. Mice also have lower visual acuity than rats\cite{Prusky2000}, and rats can be found more often during the daylight when searching for food, evidence that they are not strictly nocturnal. On the flip-side, tools for genetic access are quickly catching up in rats, making rats and increasingly powerful model in which a rich history of cognitive and other behaviors can be combined with methods for circuit dissection and mechanistic understanding. 

% Given likely differences in response characterizations between optical imaging and and electrophysiology approaches\cite{SiegleReconcilingElectrophysiology}, it is challenging to interpret results and attribute possible differences to methods as opposed to species-specific adaptations. 

% Even among rodents, there may be species-specific differences in how analogous brain areas are to parts of the primate ventral stream. For example, in mice, areas LM and AL seem to contain the most object-specific, view-tolerant information\cite{Froudarakis2020}, while studies in rats point to areas LI, LL, or even TO\cite{Vermaerke2014, Tafazoli2017}. Such differences may be due to behavioral differences in visual capacities between rats and mice, \textit{e.g.}, visual acuity\cite{Prusky2000}, how diurnal or nocturnal they are, and the extent to which they rely on visually-guided behavior\cite{Whishaw1995, Whishaw1996}. Moreover, even among rat studies, paradigms differ in stimulus set, recording techniques, and behavior states. 

\section{Goals of the present study}
% PRESENT STUDY -- what am i doing and why.

Rats have been known to be capable of visual shape discrimination for more than 100 years --- however, the only area of rat visual cortex to be imaged with cellular resolution has been V1\cite{Ohki2005,Greenberg2008}. Few or no paradigms exist for large-scale cellular resolution imaging of extrastriate cortex in awake, head-fixed rats. Furthermore, the lateral position of candidate object recognition areas (in rats, areas V1, LM, LI, LL, and TE) makes optical access all the more challenging. Nonetheless, head-fixed preparations afford certain experimental advantages, such as restrained movement, precise stimulus control, and longitudinal studies that can track the same neurons over long timescales. 

The overarching goal of the present study to bridge the gap between the long history of cognitive behaviors in rats and the more recent successes of genetic models for circuit dissection. I demonstrate reliable, high-throughput cellular resolution optical imaging in awake, head-fixed rats for the first time, and show the feasibility of applying modern imaging approaches to a powerful model of cognition. Specifically, I focus on visual perception. Although first described in rats >100 years ago, and many times since then, visual object recognition is not tested in the same high-throughput capacity in rats as other cognitive behaviors, particularly, as done with decision-making and working memory tasks. In addition, although recent studies have begun to explore visual object response properties in rat extrastriate cortex, no approaches exist to access these same areas with cellular resolution imaging methods. 

In Chapter 1, I develop a system for automated, high-throughput training of visual behaviors in rats. This work builds off of previous work on visual object recognition in rats, and is inspired by the many high-throughput behavior systems developed for testing non-vision-specific tasks in rodents. I describe an open-source, modular, and high-throughput system in which large cohorts of rats can be trained on complex visual behaviors. In this chapter, I also describe rats' perceptual choices in response to visual stimuli for which I will characterize neural responses in later chapters. In Chapter 2, I describe new systems for optical imaging in head-fixed rats, starting from optimized surgical techniques, to areal identification, and custom approaches that have allowed us to reliably image many awake, head-fixed rats in visual experiments. Since this work is the first time rat lateral cortex has been studied with cellular resolution imaging, in Chapter 3, I present a systematic survey of basic response properties in several visual areas. I describe the macro- and micro-scale organization of these areas in the context of a range of stimulus classes. In Chapter 4, I return to the object stimuli used to test the visual behavior of trained rats. Here, I examine how neural populations in different visual cortical areas respond to feature variations that either preserve or alter object identity. Finally, I conclude with a discussion on the developments and findings of the previous chapters, placing them in the context of previous findings, while considering future work that can build from the present study.

In summary, the work described in the following chapters aims to fill gaps in both technical and biological knowledge in a rat model of visual perception. The technical developments offer the potential for a new, powerful, complementary animal model for the study of high-level vision, while also opening new directions for the study of many other cognitive behaviors in rats. The scientific developments provide a basic foundation for future studies of rat visual cortex, in the hopes that with many technical barriers now overcome, the rat can be continue to be leveraged as a powerful model for mechanisms of cognition. 


%%%% 

% With the continuing development of technology for better experiment systems and improved access to brains in behaving animals, it is increasingly possible to study neural processes related to cognitive function in diverse species, from ants\cite{Kronauer} and flies\cite{Jayaraman}, to X, and even cephalopods\cite{}. However, this has only become possible in recent years --- even in more traditional animal models of cognition, insight into mechanistic understanding lags far behind the discovery and parameterization of behavioral and neural phenomena.


% Vision is critical for guiding behavior in species throughout the animal kingdom. Though nonhuman primates have been the classic model for vision, alternative systems offer a way to reveal general principles of visual processing as well as species-specific adaptations. Advances in imaging techniques and genetic tools have made genetically accessible animals, like mice, valuable systems for understanding neural circuits of vision. In contrast, although rats have long been studied for visually-guided behaviors, large-scale cellular resolution access to their visual cortex has proved to be far more challenging. Here, we aim to bridge the gap between neural circuit access and rich behavioral capacities in the rat. We characterize the functional organization of a subset of visual areas for the first time with cellular resolution imaging in awake, head-fixed animals. 


% REFREF WHY head fixed


% TABLE, prev studies
% \begin{tabular}{ |p{3cm}|p{3cm}|p{3cm}|  }
% \hline
% \multicolumn{3}{|c|}{Country List} \\
% \hline
% Country Name     or Area Name& ISO ALPHA 2 Code &ISO ALPHA 3 \\
% \hline
% Afghanistan & AF &AFG \\
% Aland Islands & AX   & ALA \\
% Albania &AL & ALB \\
% Algeria    &DZ & DZA \\
% American Samoa & AS & ASM \\
% Andorra & AD & AND   \\
% Angola & AO & AGO \\
% \hline
% \end{tabular}


